% Options for packages loaded elsewhere
\PassOptionsToPackage{unicode}{hyperref}
\PassOptionsToPackage{hyphens}{url}
%
\documentclass[
]{article}
\title{P3 Data Enrich/Format/Blend (Revised)}
\author{Team 1: Adrita Anika, Anantha Sai Sreelekha, Soham Sarda}
\date{2022-03-25}

\usepackage{amsmath,amssymb}
\usepackage{lmodern}
\usepackage{iftex}
\ifPDFTeX
  \usepackage[T1]{fontenc}
  \usepackage[utf8]{inputenc}
  \usepackage{textcomp} % provide euro and other symbols
\else % if luatex or xetex
  \usepackage{unicode-math}
  \defaultfontfeatures{Scale=MatchLowercase}
  \defaultfontfeatures[\rmfamily]{Ligatures=TeX,Scale=1}
\fi
% Use upquote if available, for straight quotes in verbatim environments
\IfFileExists{upquote.sty}{\usepackage{upquote}}{}
\IfFileExists{microtype.sty}{% use microtype if available
  \usepackage[]{microtype}
  \UseMicrotypeSet[protrusion]{basicmath} % disable protrusion for tt fonts
}{}
\makeatletter
\@ifundefined{KOMAClassName}{% if non-KOMA class
  \IfFileExists{parskip.sty}{%
    \usepackage{parskip}
  }{% else
    \setlength{\parindent}{0pt}
    \setlength{\parskip}{6pt plus 2pt minus 1pt}}
}{% if KOMA class
  \KOMAoptions{parskip=half}}
\makeatother
\usepackage{xcolor}
\IfFileExists{xurl.sty}{\usepackage{xurl}}{} % add URL line breaks if available
\IfFileExists{bookmark.sty}{\usepackage{bookmark}}{\usepackage{hyperref}}
\hypersetup{
  pdftitle={P3 Data Enrich/Format/Blend (Revised)},
  pdfauthor={Team 1: Adrita Anika, Anantha Sai Sreelekha, Soham Sarda},
  hidelinks,
  pdfcreator={LaTeX via pandoc}}
\urlstyle{same} % disable monospaced font for URLs
\usepackage[margin=1in]{geometry}
\usepackage{color}
\usepackage{fancyvrb}
\newcommand{\VerbBar}{|}
\newcommand{\VERB}{\Verb[commandchars=\\\{\}]}
\DefineVerbatimEnvironment{Highlighting}{Verbatim}{commandchars=\\\{\}}
% Add ',fontsize=\small' for more characters per line
\usepackage{framed}
\definecolor{shadecolor}{RGB}{248,248,248}
\newenvironment{Shaded}{\begin{snugshade}}{\end{snugshade}}
\newcommand{\AlertTok}[1]{\textcolor[rgb]{0.94,0.16,0.16}{#1}}
\newcommand{\AnnotationTok}[1]{\textcolor[rgb]{0.56,0.35,0.01}{\textbf{\textit{#1}}}}
\newcommand{\AttributeTok}[1]{\textcolor[rgb]{0.77,0.63,0.00}{#1}}
\newcommand{\BaseNTok}[1]{\textcolor[rgb]{0.00,0.00,0.81}{#1}}
\newcommand{\BuiltInTok}[1]{#1}
\newcommand{\CharTok}[1]{\textcolor[rgb]{0.31,0.60,0.02}{#1}}
\newcommand{\CommentTok}[1]{\textcolor[rgb]{0.56,0.35,0.01}{\textit{#1}}}
\newcommand{\CommentVarTok}[1]{\textcolor[rgb]{0.56,0.35,0.01}{\textbf{\textit{#1}}}}
\newcommand{\ConstantTok}[1]{\textcolor[rgb]{0.00,0.00,0.00}{#1}}
\newcommand{\ControlFlowTok}[1]{\textcolor[rgb]{0.13,0.29,0.53}{\textbf{#1}}}
\newcommand{\DataTypeTok}[1]{\textcolor[rgb]{0.13,0.29,0.53}{#1}}
\newcommand{\DecValTok}[1]{\textcolor[rgb]{0.00,0.00,0.81}{#1}}
\newcommand{\DocumentationTok}[1]{\textcolor[rgb]{0.56,0.35,0.01}{\textbf{\textit{#1}}}}
\newcommand{\ErrorTok}[1]{\textcolor[rgb]{0.64,0.00,0.00}{\textbf{#1}}}
\newcommand{\ExtensionTok}[1]{#1}
\newcommand{\FloatTok}[1]{\textcolor[rgb]{0.00,0.00,0.81}{#1}}
\newcommand{\FunctionTok}[1]{\textcolor[rgb]{0.00,0.00,0.00}{#1}}
\newcommand{\ImportTok}[1]{#1}
\newcommand{\InformationTok}[1]{\textcolor[rgb]{0.56,0.35,0.01}{\textbf{\textit{#1}}}}
\newcommand{\KeywordTok}[1]{\textcolor[rgb]{0.13,0.29,0.53}{\textbf{#1}}}
\newcommand{\NormalTok}[1]{#1}
\newcommand{\OperatorTok}[1]{\textcolor[rgb]{0.81,0.36,0.00}{\textbf{#1}}}
\newcommand{\OtherTok}[1]{\textcolor[rgb]{0.56,0.35,0.01}{#1}}
\newcommand{\PreprocessorTok}[1]{\textcolor[rgb]{0.56,0.35,0.01}{\textit{#1}}}
\newcommand{\RegionMarkerTok}[1]{#1}
\newcommand{\SpecialCharTok}[1]{\textcolor[rgb]{0.00,0.00,0.00}{#1}}
\newcommand{\SpecialStringTok}[1]{\textcolor[rgb]{0.31,0.60,0.02}{#1}}
\newcommand{\StringTok}[1]{\textcolor[rgb]{0.31,0.60,0.02}{#1}}
\newcommand{\VariableTok}[1]{\textcolor[rgb]{0.00,0.00,0.00}{#1}}
\newcommand{\VerbatimStringTok}[1]{\textcolor[rgb]{0.31,0.60,0.02}{#1}}
\newcommand{\WarningTok}[1]{\textcolor[rgb]{0.56,0.35,0.01}{\textbf{\textit{#1}}}}
\usepackage{graphicx}
\makeatletter
\def\maxwidth{\ifdim\Gin@nat@width>\linewidth\linewidth\else\Gin@nat@width\fi}
\def\maxheight{\ifdim\Gin@nat@height>\textheight\textheight\else\Gin@nat@height\fi}
\makeatother
% Scale images if necessary, so that they will not overflow the page
% margins by default, and it is still possible to overwrite the defaults
% using explicit options in \includegraphics[width, height, ...]{}
\setkeys{Gin}{width=\maxwidth,height=\maxheight,keepaspectratio}
% Set default figure placement to htbp
\makeatletter
\def\fps@figure{htbp}
\makeatother
\setlength{\emergencystretch}{3em} % prevent overfull lines
\providecommand{\tightlist}{%
  \setlength{\itemsep}{0pt}\setlength{\parskip}{0pt}}
\setcounter{secnumdepth}{-\maxdimen} % remove section numbering
\ifLuaTeX
  \usepackage{selnolig}  % disable illegal ligatures
\fi

\begin{document}
\maketitle

We have four datasets: Collision, Person, Vehicles and Weather datsets.
The following steps are taken for enrichment, blending and blending the
datasets.

\hypertarget{enrichment-and-formatting}{%
\subparagraph{Enrichment and
Formatting:}\label{enrichment-and-formatting}}

\begin{enumerate}
\def\labelenumi{\arabic{enumi}.}
\item
  Creating time-based features: We have one variable named as
  ``CRASH\_DATE' which contains the date, month and year information.
  From that, Crash Year', `Crash Month', `Crash Day', and `Time of day'
  variables were derived. These variables will be used to visualize the
  temporal distribution of crashes and look for patterns, if any.

\begin{Shaded}
\begin{Highlighting}[]
\FunctionTok{library}\NormalTok{(lubridate)}
\NormalTok{crash\_data }\OtherTok{\textless{}{-}} \FunctionTok{read.csv}\NormalTok{(}\StringTok{"NYC\_crashes.csv"}\NormalTok{)}
\NormalTok{crash\_data}\SpecialCharTok{$}\NormalTok{CRASH\_DAY }\OtherTok{\textless{}{-}} \FunctionTok{day}\NormalTok{(crash\_data}\SpecialCharTok{$}\NormalTok{CRASH\_DATE)}
\NormalTok{crash\_data}\SpecialCharTok{$}\NormalTok{CRASH\_MONTH }\OtherTok{\textless{}{-}} \FunctionTok{month}\NormalTok{(crash\_data}\SpecialCharTok{$}\NormalTok{CRASH\_DATE)}
\NormalTok{crash\_data}\SpecialCharTok{$}\NormalTok{CRASH\_YEAR }\OtherTok{\textless{}{-}} \FunctionTok{year}\NormalTok{(crash\_data}\SpecialCharTok{$}\NormalTok{CRASH\_DATE)}
\end{Highlighting}
\end{Shaded}
\item
  We want to see patterns based on weekdays (Sat, Sun,.. etc). Hence, we
  enhance the data with another variable containing the weekday
  information.

\begin{Shaded}
\begin{Highlighting}[]
\NormalTok{crash\_data}\SpecialCharTok{$}\NormalTok{CRASH\_WDAY }\OtherTok{\textless{}{-}} \FunctionTok{weekdays}\NormalTok{(}\FunctionTok{as.Date}\NormalTok{(crash\_data}\SpecialCharTok{$}\NormalTok{CRASH\_DATE))}
\end{Highlighting}
\end{Shaded}
\item
  For time analysis, we divided 24 hours of the day time into 12
  timeslots. Like: 00:00-01:59, 2:00-03:59,\ldots, 22:00-23:59. Later we
  represented this with numbers starting from 0 to 11 as shown later in
  this report.
\item
  We have the geographic information: longitude and latitude and zip
  code. We will consider the zip codes as location IDs in our analysis
  and place a marker on the map for each zip code, so we need to fix a
  longitude and latitude value for the marker to be placed for each zip
  code. We considered the average of all available longitudes under a
  zip code for placing a marker on the map. Same thing is done for
  latitude as well. This data is required for spatial analysis, we save
  this as ``cluster\_data.csv'.

\begin{Shaded}
\begin{Highlighting}[]
\FunctionTok{library}\NormalTok{(dplyr)}
\NormalTok{cluster\_data }\OtherTok{\textless{}{-}}\NormalTok{ crash\_data }\SpecialCharTok{\%\textgreater{}\%} \FunctionTok{group\_by}\NormalTok{(ZIP\_CODE) }\SpecialCharTok{\%\textgreater{}\%} \FunctionTok{summarise}\NormalTok{(}\AttributeTok{LAT =} \FunctionTok{mean}\NormalTok{(LATITUDE),}
                                                              \AttributeTok{LNG =} \FunctionTok{mean}\NormalTok{(LONGITUDE),}
                                                              \AttributeTok{Crash =} \FunctionTok{n}\NormalTok{(),)}
\end{Highlighting}
\end{Shaded}
\item
  All the column headers are once again formatted with a consistency
  followed - Upper case letters with underscores between words.
\item
  For our analysis we created another variable named as
  ``VEHICLE\_TYPE'' for the vehicles dataset. We categorized all the
  vehicles depending on their sizes into four categories: very\_large,
  large, medium, small. This has been done following this paper
  {[}\href{https://www.researchgate.net/publication/337243001_Applying_Big_Data_Analytics_on_Motor_Vehicle_Collision_Predictions_in_New_York_City}{Paper
  Link}{]}
\item
  The weather dataset does not have any common variable apart from
  geodetic reference. The Lat-Long coordinates were used to find the
  nearest weather station to each crash activity, and was then the
  weather data for that station on the particular day of crash was
  linked to the crash data.
\item
  We are considering a classification problem of traffic collision
  severity based on ten features: emotional status, driver license
  status, vehicle type, two contributing factors, position in vehicle,
  safety equipment, day and time. We converted some of these variables
  into numerical values with one hot encoding. This is done only for the
  classification problem. Following code chunk shows the one-hot
  encoding process for only one variable: position in vehicle.

\begin{Shaded}
\begin{Highlighting}[]
\NormalTok{perX }\OtherTok{\textless{}{-}} \FunctionTok{read.csv}\NormalTok{(}\StringTok{"NYC\_persons.csv"}\NormalTok{)}
\NormalTok{per }\OtherTok{\textless{}{-}}\NormalTok{ perX[, }\FunctionTok{c}\NormalTok{(}\StringTok{"COLLISION\_ID"}\NormalTok{,}\StringTok{"POSITION\_IN\_VEHICLE"}\NormalTok{)]}
\NormalTok{per }\OtherTok{\textless{}{-}} \FunctionTok{filter}\NormalTok{(per, per}\SpecialCharTok{$}\NormalTok{POSITION\_IN\_VEHICLE}\SpecialCharTok{!=}\StringTok{"Unknown"}\NormalTok{)}
\FunctionTok{library}\NormalTok{(caret)}
\CommentTok{\#define one{-}hot encoding function}
\NormalTok{dummy }\OtherTok{\textless{}{-}} \FunctionTok{dummyVars}\NormalTok{(}\StringTok{" \textasciitilde{} ."}\NormalTok{, }\AttributeTok{data=}\NormalTok{per)}
\CommentTok{\#perform one{-}hot encoding on data frame}
\NormalTok{final\_df }\OtherTok{\textless{}{-}} \FunctionTok{data.frame}\NormalTok{(}\FunctionTok{predict}\NormalTok{(dummy, }\AttributeTok{newdata=}\NormalTok{per))}
\end{Highlighting}
\end{Shaded}
\end{enumerate}

\hypertarget{mergingblending}{%
\paragraph{Merging/Blending:}\label{mergingblending}}

\begin{enumerate}
\def\labelenumi{\arabic{enumi}.}
\item
  All of the files will be stored in csv format. As we have huge
  datasets, we created seperate csv files for some of the analysis. For
  the classification of Traffic collision based on severity into three
  classes (No hurt, Injury, Lethal) we will use ten attributes from the
  three datasets: collision, person and vehicle. All these datasets
  contain one common attribute ``COLLISION ID''. We inner joined these
  datasets and kept the ten required features. First few rows of the
  dataset is shown below. The result is presented after doing one hot
  encoding.

\begin{Shaded}
\begin{Highlighting}[]
\NormalTok{clf\_data }\OtherTok{\textless{}{-}} \FunctionTok{read.csv}\NormalTok{(}\StringTok{"dfx.csv"}\NormalTok{)}
\FunctionTok{head}\NormalTok{(clf\_data, }\DecValTok{3}\NormalTok{)}
\end{Highlighting}
\end{Shaded}

\begin{verbatim}
##   X CONTRIBUTING_FACTOR_VEHICLE_1 EMOTIONAL_STATUS POSITION_IN_VEHICLE
## 1 1                            15                7                   4
## 2 2                            15                5                   2
## 3 3                            15                5                   2
##   SAFETY_EQUIPMENT CRASH_TIME VEHICLE_TYPE DRIVER_SEX DRIVER_LICENSE_STATUS
## 1                8      22:20          178          2                     1
## 2                2      22:20          178          1                     0
## 3                2      22:20          178          2                     1
##   CONTRIBUTING_FACTOR_1 Day label
## 1                    28   2     1
## 2                    15   2     1
## 3                    28   2     1
\end{verbatim}
\item
  As mentionied earlier, ``cluster\_data.csv'' wil be used for spatial
  analysis for finding collision hotspots.

\begin{Shaded}
\begin{Highlighting}[]
\FunctionTok{head}\NormalTok{(cluster\_data,}\DecValTok{3}\NormalTok{)}
\end{Highlighting}
\end{Shaded}

\begin{verbatim}
## # A tibble: 3 x 4
##   ZIP_CODE   LAT   LNG Crash
##      <int> <dbl> <dbl> <int>
## 1    10001  40.8 -74.0     1
## 2    10012  40.7 -74.0     1
## 3    10016  40.7 -74.0     3
\end{verbatim}
\item
  For time analysis, we created another csv file that has the
  information of weekdays and timeslot of the day and corresponding
  number of collisions. We need this is data for two analysis (
  forecasting and visual analysis)

\begin{Shaded}
\begin{Highlighting}[]
\NormalTok{time\_data }\OtherTok{\textless{}{-}} \FunctionTok{read.csv}\NormalTok{(}\StringTok{"refined\_time.csv"}\NormalTok{) }
\FunctionTok{head}\NormalTok{(time\_data,}\DecValTok{3}\NormalTok{)}
\end{Highlighting}
\end{Shaded}

\begin{verbatim}
##      day new_time numCrash   time_slot
## 1 Friday        0      430 00:00-01:59
## 2 Friday        1      150 02:00-03:59
## 3 Friday        2      164 04:00-05:59
\end{verbatim}
\item
  As mentioned earlier, we categoried vehicles into four groups. The
  dataset is shown below:

\begin{Shaded}
\begin{Highlighting}[]
\NormalTok{vehicle\_data }\OtherTok{\textless{}{-}} \FunctionTok{read.csv}\NormalTok{(}\StringTok{"veh.csv"}\NormalTok{)}
\FunctionTok{head}\NormalTok{(vehicle\_data,}\DecValTok{3}\NormalTok{)}
\end{Highlighting}
\end{Shaded}

\begin{verbatim}
##        X  BOROUGH NUMBER_OF_PERSONS_INJURED NUMBER_OF_PERSONS_KILLED
## 1 126446 BROOKLYN                         0                        0
## 2 126447   QUEENS                         0                        0
## 3 126448 BROOKLYN                         0                        0
##   NUMBER_OF_PEDESTRIANS_INJURED NUMBER_OF_PEDESTRIANS_KILLED
## 1                             0                            0
## 2                             0                            0
## 3                             0                            0
##   NUMBER_OF_CYCLIST_INJURED NUMBER_OF_CYCLIST_KILLED NUMBER_OF_MOTORIST_INJURED
## 1                         0                        0                          0
## 2                         0                        0                          0
## 3                         0                        0                          0
##   NUMBER_OF_MOTORIST_KILLED                 VEHICLE_TYPE_CODE_1   TYPE
## 1                         0 Station Wagon/Sport Utility Vehicle  small
## 2                         0                               Sedan  small
## 3                         0                       Pick-up Truck medium
\end{verbatim}
\item
  Finally, our weather dataset is shown:
\end{enumerate}

\begin{Shaded}
\begin{Highlighting}[]
\NormalTok{weather\_data }\OtherTok{\textless{}{-}} \FunctionTok{read.csv}\NormalTok{(}\StringTok{"weather.csv"}\NormalTok{)}
\FunctionTok{head}\NormalTok{(weather\_data, }\DecValTok{3}\NormalTok{)}
\end{Highlighting}
\end{Shaded}

\begin{verbatim}
##       X COLLISION_ID       DATE  TIME   BOROUGH    lat_x     lon_x
## 1 37250      4273965 2020-01-14 18:10    QUEENS 40.70177 -73.90070
## 2 69952      4128277 2019-05-08 14:24 MANHATTAN 40.76617 -73.95425
## 3 69963      4128012 2019-05-08 10:30 MANHATTAN 40.75931 -73.96530
##                LOCATION NUMBER_OF_PERSONS_INJURED NUMBER_OF_PERSONS_KILLED
## 1    40.70177, -73.9007                         0                        0
## 2 40.766167, -73.954254                         0                        0
## 3   40.759308, -73.9653                         0                        0
##   NUMBER_OF_CYCLIST_INJURED NUMBER_OF_CYCLIST_KILLED NUMBER_OF_MOTORIST_INJURED
## 1                         0                        0                          0
## 2                         0                        0                          0
## 3                         0                        0                          0
##   NUMBER_OF_MOTORIST_KILLED  CONTRIBUTING_FACTOR_VEHICLE_1 VEHICLE_TYPE_CODE_1
## 1                         0          Following Too Closely               Sedan
## 2                         0 Driver Inattention/Distraction               Sedan
## 3                         0 Driver Inattention/Distraction                 Bus
##   VEHICLE_TYPE_CODE_2 YEAR MONTH                        NAME PRCP SNOW SNWD
## 1               Sedan 2020     1 NY CITY CENTRAL PARK, NY US 0.07    0    0
## 2               Sedan 2019     5 NY CITY CENTRAL PARK, NY US 0.00    0    0
## 3           Box Truck 2019     5 NY CITY CENTRAL PARK, NY US 0.00    0    0
\end{verbatim}

\end{document}
